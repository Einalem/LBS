\section{Location based Services in der Praxis}
	\subsection{Anwendungsbereiche}
Location Based Services, also mobile, positionsbezogene Dienste haben allgemein ein sehr breites Einsatzgebiet.\\
Der Autoren Allan J Brimicombe und Chao Li unterscheiden in ihrem Buch "'Location-Based Services and Geo-Information Engineering"' ~\cite[S.132]{brimicombe_li:application_area} zehn verschiedene Einsatzgebiete:
\begin{itemize}
	\item Navigation\\
Navigation ist die gezielte Führung des Nutzers von Punkt A nach Punkt B. Manche Geräte bieten auch eine Echtzeit-Analyse an.
	\item Wegfindung\\
Bei der Wegfindung hingegen liegt der Fokus auf dem Finden möglicher Wege, d.h. sie dient der allgemeinen Orientierung des Nutzers.
	\item Echtzeit-Verfolgung\\
Verfolgungs- auch Tracking-Systeme genannt, dienen der Echtzeitanalyse des Nutzerstandorts, um diesem z.B. das Finden von Freunden in der näheren Umgebung zu erleichtern.
	\item Elektronischer Handel\\
Bei Anwendungen aus dem Bereich des elektronischen Handel, auch E-Commerce genannt, handelt es sich um werbende Produkte, die dem Nutzer auf Basis seiner Position ortsspezifische Angebote eröffnen.
	\item User-solicited Informations (vom Nutzer gewünschte Informationen)\\
Unter diese Kategorie fallen alle Anwendungen, die vom Nutzer für den geschäftlichen oder sozialen Gebrauch genutzt werden. Beispiele dafür sind: Wetterprognosen, Zugverspätungen und Filmvorführungen.
	\item Ortsgebundene Tarife
	\item Fulfilment
	\item Koordination
	\item Kunstvoller Ausdruck
	\item Mobile Spiele
\end{itemize}

Anwendungsbereiche/Einsatzgebiete von LBS
		\subsubsection{Praxis Teil 1}
...
	\subsection{Typen von Location based Services (proaktiv und reaktiv)}
Typen
	\subsubsection{Typen Teil 1}
...
	\subsection{Location based Services auf mobilen Endgeräten}
Beispiele
	\subsubsection{Aufzählen vieler Anwendungsbeispiele mit Erläuterung des Nutzens}
...
	\subsubsection{Umsetzungsmöglichkeiten für die Beispiele nennen}
...