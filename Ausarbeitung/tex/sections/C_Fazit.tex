\newpage
\section{Fazit}


Mit dieser Studienarbeit wurden die theoretischen Grundlagen zu Location-based Services erarbeitet. 
Praxisbezogene Anwendungsfälle für LBS wurden aufgezeigt und beschrieben.

Ein Anwendungsfall würde in dieser Studienarbeit als Smartphone-App prototypisch umgesetzt. 
Es wurde dabei gezeigt, dass eine Umsetzung von LBS mit den Technologien Apache Cordova, Ionic, HTML5, CSS und Angular-JS möglich ist. 

Mit der Entscheidung für diese Technologien, sind einige Vorteile entstanden. Hierzu zählt, dass die Vorkenntnisse aus diversen Vorlesungen genutzt werden konnten und somit eine schnelle Einarbeitung in das Thema mobile Applikation stattfand. Des Weiteren ist es gelungen, eine App zu entwickeln, welche wie eine native App aussieht. Von Vorteil war auch die Nutzung der Hardware des Smartphones durch Plug-Ins des Cordova Frameworks. Als Nachteil ist zu nennen, dass der Aufwand für die Installation und Aufsetzung der App hoch war.

Im Detail wurde auf die Nutzung von externem Kartenmaterial und die Erweiterung der Positionsbestimmung um iBeacons umgesetzt.
Resultierend daraus wurde festgestellt, dass sich Openstreetmap gut für studentische Projekte eignet. Des Weiteren konnte mit der Erweiterung der Positionsbestimmung um iBeacons eine deutlich verbesserte Präzision erzielt werden. 

Abschließend ist zu nennen, dass die Smartphone-App erfolgreich umgesetzt wurde und damit gezeigt wurde, dass die Technologien Apache Cordova, Ionic, HTML5, CSS und Angular-JS für LBS eignen.


\subsection[Ausblick]{Ausblick
 \\ \textnormal{\small{\textit {Verfasst von ??}}}}
 
Es ist jetzt eine App vorhanden, welche viele Dinge prototypisch enthält:
\\Positionsbestimmung via GPS
\\Kartenmatrial
\\iBeacons...
Damit sind viele Grundfunktionalitäten vorhanden um die App weiter aus- und umzubauen.
Weiterer Aspekt, mit den richtigen Mitteln könnte man die Plattformunabhängigkeit testen iOS...