\begin{abstract}

Im Rahmen dieser Studienarbeit wird zunächst der Begriff eines Location-based Service definiert und auf spezifische Technologien in diesem Zusammenhang eingegangen. Als Verfahren zur Positionsbestimmung werden Satellitenpositionierung, Positionierung in Mobilfunknetzen und Positionsbestimmung in Gebäuden betrachtet. Anschließend werden Anwendungsbereiche und Nutzer vorgestellt. Den praktischen Teil bildet die prototypische Umsetzung einer Location-based Services - Applikation. Bei der Entwicklung werden unter Anderem die Frameworks Cordova, Ionic und Angular-JS eingesetzt.
Das Fazit dieser Arbeit ist, dass Location-based Services eine große Zukunft haben und schon der einfach gehaltene Prototyp mit den gewählten und leicht zu erlernenden Technologien erfolgreich umgesetzt werden kann. Die Präzision der Positionsbestimmung kann mithilfe der iBeacon-Technik weiter erhöht werden.
\end{abstract}