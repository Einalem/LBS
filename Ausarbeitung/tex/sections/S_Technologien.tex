\subsection{Technologien und Entscheidungen}

\subsubsection{Cordova Phonegap}

\subsubsection{HTML5}

\subsubsection{CSS}

\subsubsection{JS}

\subsubsection{Kartenmaterial}

Kartenmaterial im Browser bzw. der Hybrid-App ist ein essentieller Bestandteil von Location based Services. Durch eine Positionsbestimmung alleine erhält man nur Daten die für den Nutzer nicht anschaulich sind. Diese liegen normalerweise als geografische Koordinaten vor, die in geografischer Breite und geografischer Länge angegeben werden. Eine Beispielposition soll die Bedeutung von Kartenmaterial für den Nutzer von Location based Services verdeutlichen.

Als Beispiel hierfür wurde die Position der DHBW Mannheim in der Coblitzallee gewählt. Hierbei werden die geografischen Koordinaten, eine Adresse und ein Kartenausschnitt in einer Tabelle gegenübergestellt. Siehe hierzu Tabelle \ref{BedeutungVonKartenmaterial}.

\begin{table}[htbp]
\begin{center}
\begin{tabular}{|p{4.75cm}p{4.75cm}p{4.75cm}|} 
	\hline
		\rowcolor{black} \textcolor{white} { \textbf{Geographische Koordinaten} } & \textcolor{white}{\textbf{Adresse}} & \textcolor{white}{\textbf{Kartenausschnitt}}\\ 
		\rowcolor[gray]{.75}  49$^\circ$28'27.6\grqq N 8$^\circ$32'03.9\grqq E & Duale Hochschule Baden-Württemberg Mannheim \newline 
Coblitzallee 1-9 \newline 
68163 Mannheim \newline (Neuostheim) & Kartenausschnitt\newline 
\includegraphics[width=0.3\textwidth]{ref/images/KartenmaterialKlein.png} \\ 
\hline
	\end{tabular}
\end{center}
\caption{Bedeutung von Kartenmaterial} \label{BedeutungVonKartenmaterial}
\end{table}

In der Tabelle sind verschiedenen Ortsdaten zur Verfügung gestellt, die alle Vor- und Nachteile aufweisen.

Die Geographischen Koordinaten geben die Position am genausten an, sind für fast keine Nutzer einer App von Bedeutung. 

Die Adresse ist im Alltag am geläufigsten und somit für Nutzer am verständlichsten. Allerdings ist die Angabe nicht so genau, wie die Geographischen Koordinaten. Denn die Angabe Hausnummer 1-9 gibt einen relativ großen Bereich an.

Die Vorteile eines Kartenausschnitts sind, dass die Detaillierung vom Nutzer angepasst werden kann. Des Weiteren werden viele grafische Informationen angezeigt, wie zum Beispiel der eigene Standort, an denen sich ein Nutzer Orientieren kann. Der Nachteil dieser Variante ist, dass die Kartenausschnitte die Zuhilfenahme von externen Quellen und einem erhöhten TODO: Programmieraufwand mit sich bringen.


TODO: Quelle finden
Auf Smartphones gehört Kartenmaterial und dessen Integration in Apps mittlerweile zum Standard, an welchen sich Nutzer gewöhnt haben. Aus diesem Grund sollte auch Kartenmaterial in die Location based Services App integriert werden, welche die Autoren bei dieser Studienarbeit entwickeln. 

Mögliche Quellen für das Kartenmaterial sind \glqq Google Maps\grqq, \glqq Bing Maps\grqq  und \glqq Open Street Maps\grqq.


\textbf{Anforderungen an das Kartenmaterial}

Die Anforderungen an interaktives Kartenmaterial bezüglich der in dieser Studienarbeit entwickeltem Projekt lassen sich in zwei Gruppen einteilen, die funktionalen und nichtfunktionalen Anforderungen.

Die \underline{nichtfunktionalen Anforderungen} sind:
\begin{enumerate}
\item Kostenlose Abfragen
\item Ohne Account nutzbar
\item Gute Dokumentation mit Codebeispielen
\item Zukunftssicherheit
\end{enumerate}

Die \underline{funktionalen Anforderungen} an das interaktive Kartenmaterial sind:
\begin{enumerate}
\item JavaScript API
\item Unterstütze Browser
\item Eigenen Standort anzeigen
\item Markierungen auf der Karte setzen
\item Markierungen bündeln (optional)
\end{enumerate}

Bevor \glqq Google Maps\grqq, \glqq Bing Maps\grqq  und \glqq Open Street Maps\grqq bezüglich der Anforderungen untersucht werden, müssen diese genauer spezifiziert werden.

Zuerst widmen wir uns den nichtfunktionalen Anforderungen.
\begin{enumerate}
\item Kostenlose Abfragen \\
Da es sich bei der Implementierung um einen Prototypen für diese Studienarbeit handelt und dieser nicht kommerziell verwendet werden soll, sollen auch die Abfragen (map-loads) kostenlos sein. Zudem sollten genug kostenlose Abfragen zur Verfügung stehen. Bei 3 Entwicklern und Tests über die Dauer der Studienarbeit (8-9 Monate) darf das Kontingent der kostenlosen Abfragen nicht aufgebraucht sein.

\item Ohne Account nutzbar\\
Die Nutzung ohne Account vereinfacht den Einstieg für das Kartenmaterial und sollte deshalb gewährleistet sein. Zudem würden dann alle Teammitglieder dieser Studienarbeit mit dem selben Account eines Teammitglieds arbeiten. 

\item Gute Dokumentation mit Codebeispielen\\
Eine gute Dokumentation der API des Kartenmaterials mit vielen Codebeispielen erleichtert den Einstieg deutlich. Da alle Teammitglieder dieser Studienarbeit noch keine Erfahrung mit Kartenmaterial haben ist dies eine wichtige Anforderung. 

\item Zukunftssicherheit\\
Da diese Arbeit nicht nur den aktuellen Stand der Technik abbilden soll, sondern auch einen Ausblick geben soll ist es wichtig, dass auch beim Kartenmaterial auf die Zukunftssicherheit geachtet wird.
\end{enumerate}



TODO: Überleitung

\begin{enumerate}
\item JavaScript API\\
Eine JavaScript API ist essentiell wichtig, da der Prototyp mit HTML, CSS und Java Script entwickelt wird. 

\item Unterstütze Browser\\
Android ??, IOS, 

\item Eigenen Standort anzeigen\\
Der eigene Standort muss grafisch auf einer interaktiven Karte angezeigt werden. Das zentrieren des Kartenmaterials auf den eigenen Standort soll auch möglich sein.

\item Markierungen auf der Karte setzen\\
Eigenen Markierungen müssen auf der Karte gesetzt werden könne. Dies muss grafisch erfolgen, denn es ist für den Prototypen besonders wichtig zu veranschaulichen, wo sich das gesuchte Ziel befindet. 

\item Markierungen bündeln (optional)\\
Markierungen auf der Karte sollten gebündelt werden, wenn der Zoom-Faktor zu klein wird. Dies soll der Übersichtlichkeit bei vielen Markierungen auf der Karte dienen. Hierbei soll der Radius, in dem Markierungen gebündelt werden eingestellt werden könne.

\end{enumerate}

\textbf{\underline{Google Maps}}

2500 Abfragen pro Tag frei
https://developers.google.com/maps/licensing?hl=de

ohne Key nutzbar (seit version 3)
http://googlegeodevelopers.blogspot.de/2010/03/introducing-new-google-geocoding-web.html

Gute Dokumentation mit vielen Beispielen:
fast 150 Beispiel von Google selbst
https://developers.google.com/maps/documentation/javascript/examples/?hl=de

Nicht nur Beispiele von Google 

w3schools.com
- Center Map on Position
- Add a Marker + eigenes Bild + Infowindow for Marker






\textbf{funktionale Anforderungen}

\underline{JavaScript API}

JavaScript API Version 3

<script src="https://maps.googleapis.com/maps/api/js?v=3.exp"></script>
    <script>
var map;
function initialize() {
  var mapOptions = {
    zoom: 8,
    center: new google.maps.LatLng(-34.397, 150.644)
  };
  map = new google.maps.Map(document.getElementById('map-canvas'),
      mapOptions);
}

google.maps.event.addDomListener(window, 'load', initialize);

    </script>





\textbf{\underline{Bing Maps}}

Kostenlose Abfragen für education use
Viele unterschiedliche Lizenzmodelle 
125.000 Abfragen pro Jahr

Man muss allerdings einen Bing Maps Account erstellen

http://www.microsoft.com/maps/Licensing/licensing.aspx\#StepTitle


Ausführliche Dokumentation mit vielen Beispielen
+ für in Browser Tests auf der Webseite
https://www.bingmapsportal.com/Isdk/AjaxV7\#CreateMapWithMapOptions1

https://msdn.microsoft.com/en-us/library/gg427610.aspx

\textbf{funktionale Anforderungen}
JavaScript gegeben

Supported Browsers
https://msdn.microsoft.com/en-us/library/gg427618.aspx


\textbf{\underline{Open Street Maps}}

Kostenlos und unbegrenzt viele Abfragen

ohne Account nutzbar

Gute Dokumentation mit vielen Beispielen leider nicht gegeben




