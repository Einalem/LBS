\section{Anwendungsfälle für LBS}

Aufzählen wie in SE (Nutzer Potentiale, welches Nutzerproblem wird befriedigt, wie ist Konkurrenz)

Wecker
Restaurantfinder
Navigation
Freunde finden
Reiseführer
Spiele


\subsection{Historie von LBS}
Bedeutung von Location Based Services in der heutigen Zeit. Erfunden wurden LBS von der U.S. abwehr behörde. Hierfür wurde das System Navstar entwickelt. Mit Hilfe von Sattelitten kann die Position bis auf wenige Meter bestimmt werde. Bekannt geworden ist diese System unter dem Namen Global Positioning System (GPS). Nutzer dieses System war das Militär. 1980 entschied man sich dazu, dass System der Öffentlichkeit bereitzustellen. Ziel dieses Schrittes war es, Fortschritte in der Entwicklung von Satelliten Systemen zu machen.

Die Europäische Union entwickelte daraufhin mit der Europäischen Raumfahrt Behörde (ESA) einen eigenes System namens Galileo. Galileo verwendet ähnliche Frequenzen wie GPS, würde die UN oder die USA diese Signale blockieren, wären beide Systeme gestört, was politisch von hoher Bedeutung ist.




Location Based Services, also mobile, positionsbezogene Dienste haben allgemein ein sehr breites Einsatzgebiet.\\
\paragraph{Theoretische Einsatzgebiete}
Der Autoren Allan J Brimicombe und Chao Li unterscheiden in ihrem Buch "'Location-Based Services and Geo-Information Engineering"' ~\cite[S.132]{brimicombe_li:application_area} zehn verschiedene Einsatzgebiete:
\begin{itemize}
	\item Navigation\\
The process or activity of accurately ascertaining one’s position and planning and following a route.
Seit der kostenfreien und öffentlichen Nutzung des GPS hat sich viel in der Navigationsbranche getan. Immer mehr Geräte verfügen über GPS Empfänger, welche eine Lokalisierung ermöglichen. Mit Hilfe des genauen Standortes, ist es möglich den Nutzer des Gerätes zu navigieren. Bei einer Navigation, benötigt das System einen Start- und Zielpunkt. Das Gerät berechnet eine Route und informiert den Nutzer über Distanzen und Richtungsänderungen kurz vor bevor diese ausgeführt werden müssen um der Route folgen zu können. Dies ist dem Gerät möglich, das es ständig den eigenen Standort abfragt. Folgt der Nutzer den Anweisungen befindet er sich am Ende der Navigation am Ziel.

	\item Wegfindung\\
Im Gegensatz zur Navigation wird bei der Wegfindung nur bei der Planung der Route der Standort benötigt. Eine Führung zum Ziel findet nicht statt. Im Umfeld von LBS wird bei einer Wegfindung der eigene Standort als Startpunkt gesetzt und ein Zielpunkt muss vom Nutzer angegeben werden. Ein Wegfindungsalgorithmus berechnet daraufhin eine Route. Diese kann beispielsweise auf einer Kartedargestellt werden oder jeder Richtungsänderung wird mit einer Strecke in Textform aufgelistet. Moderne Wegfindungsprogramme erlauben die zusätzliche Angabe von Routen- Kriterien. Unter anderem kann die kürzeste Strecke favorisiert werden oder die schnellste.  Hierfür benötigt das Systeme nicht nur die Streckenlänge sondern auch Tempolimits wie auf Autobahnen. 
	\item Echtzeit-Verfolgung\\
Ein weiteres Einsatzgebiet von LBS sind Verfolgungs- und Tracking-Systeme. Diese liefern in Echtzeit den Standort des Gerätes welches den Empfänger enthält. Dies kann beispielsweise genutzt werden um einen Freund in einem schwer überschaubaren Gebiet zu finden. Derjenige der gefunden werden möchte muss seinen Standort bestimmen lassen und diesen an den suchenden übermitteln. Möglich ist dies voll automatisiert durch Apps bei Smartphones. Ein ähnlicher Anwendungsfall ist die Ortung des eigenen Kindes. Das Handy des Kindes sendet in regelmäßigen Abständen die Position an eine Webseite und die Eltern können sich den Standort über eine Karte betrachten.  
	\item Elektronischer Handel\\
Im Zeitalter des Internets können viele Informationen und Aufgaben kostenlos im Internet abgerufen werden können. Einige Beispiele sind: Zeitung lesen, Recherchen durchführen, Musik hören, einkaufen. Daher gewinnt das Werbung schalten im Internet immer mehr an Bedeutung. Neben der personalisierten Werbung welche durch Nutzerdaten Erfassung ermöglicht wird, spielt der Standort des Nutzers eine Rolle zum schalten geeigneter Werbung. Neben Versand- und Online-Händlern gibt es viele Firmen welche ihre Produkte überwiegend in der Produktionsregion verkaufen für diese Händler ist Hyperlokale Werbung von großer Bedeutung. Das Verfahren ist wie folgende, zuerst wird der Standort des Nutzers ermittelt. Je nachdem auf was der Zugriff erlaubt ist kann der Standort bis auf wenige Meter bestimmt werden. Anschließend kann gezielt Werbung über das mobile Internet geschaltet werden. Diese kann auf Webseiten oder in Applikationen angezeigt werden. 
	\item User-solicited Informations (vom Nutzer gewünschte Informationen)\\
Unter diese Kategorie fallen alle Anwendungen, die vom Nutzer für den geschäftlichen oder sozialen Gebrauch genutzt werden. Beispiele dafür sind: Wetterprognosen, Zugverspätungen und Filmvorführungen.
	\item Ortsgebundene Tarife
	\item Fulfilment
	\item Koordination
	\item Kunstvoller Ausdruck
	\item Mobile Spiele
\end{itemize}


\subsection{Hauptnutzer von LBS}

LBS wurden erstmals vom amerikanischen Militär erfunden und genutzt. Nachdem die Services der Öffentlichkeit bereitgestellt wurden, führte dies zu immer mehr Anwendungsbereichen beispielsweise zur Lokalisierung von Notrufen. In Europa findet dies über die Rufnummer „112“ statt, in Amerika „911“.  Seit 1996 besteht in den USA eine Pflicht, den ungefähren Standort mitzusenden bei einem Notruf. 

Im Laufe der letzten Jahre wurden immer mehr Möglichkeiten geschaffen, mobil Telefone zu lokalisieren und den Standort für beispielsweise Informationsdarstellung zu nutzen. Damit ergibt sich der dritte große Anwendungsbereich von LBS, die kommerzielle Nutzung für privat Personen und Unternehmen.

Im folgenden Abschnitt werden die genannten Hauptnutzer von LBS genauer erläutert und aufgezeigt wofür LBS verwendet wird.


\underline{LBS im Umfeld des Militärs}

Den eigenen Standort zu kennen und beschreiben zu können ist nicht immer auf Anhieb möglich. Befindet man sich an einem Ort, bei dem viele Sehenwürdigkeiten oder bekannte Gegenstände wie beispielsweise Häuser, Parks, Straßen in der Nähe sind, fällt es einem meist einfacher den eigenen Standort einer anderen Person mittzuteilen, damit dieser einen findet.

Wesentlich schwieriger ist die Standortbestimmung wenn man sich an einem Ort befindet, der sehr allgemein ist und keine Besonderheiten bzw. identifikations Merkmale aufweist. Oft befindet sich das Militär an solchen schwer zu definierenden Orten. Einige Beispiele für solche Orte sind: Wüsten, Wälder, Berge und Gebirge. Vermutlich war dies einer der Hauptgründe ein System zur Bestimmung des Standortes zu entwickeln. 

Die Anwenudng FBCB2 ist ein Beispiel für die Verwendung von mehreren Standorten. Sie wird bereits seit 10 Jahren vom amerikanischen Militär eingesetzt. FBCB2 ist die Abkürzung für „Force-Twenty-One Battle Command Brigade and Below“. Die Anwendung ist bei Panzer Brigaden anführern im Einsatz. Auf einer Karte wird dem Nutzer angezeigt welche verbündeten Panzer in der Nähe sind. Zu jedem dieser Panzer werden einem weiter Informationen bereitgestellt. Mit diesem System braucht man keinen Kompass und keine Papier- Karte mehr um sich eine Überblick zu verschaffen. 

Im Kriegs Geschehen allgemein nimmt die Bedeutung von LBS stark zu. Genutzt wird dies Technik unter anderem bei Lenkraketen. Der Befehlshaber braucht nur die Koordinaten anzugeben und die Rakete berechnet den optimalen Weg zum Ziel. Essenziel wichtig ist dabei für die Rakete zu jedem Zeitpunkt im Flug zu wissen an welchen Standort sie sich befindet um gegebenenfalls die Geschwindigkeit oder Höhe anzupassen. 

Neben Lenkraketen gibt es immer mehr unbemannte Kriegsflugzeuge, sogenannte Drohnen. Gesteurt werden dies nicht aus dem Cockpit des Flugzeugs sondern am Boden über einen Joystick. Der Joystick ist mit einem Computer verbunden und per Kamera kann der „Pilot“ sehen wohin er fliegt. Auch in diesem Anwenungsfall ist es von sehr große Bedeutung, dass der Pilot jederzeit weiß wo er sich befindet und in welche Richtung er fliegen muss.

\underline{LBS zur Lokalisierung von Notrufen}

Benötigt man schnelle Hilfe, dann ist es von großem Vorteil, wenn der jenige, der einem Helfen soll schnellst möglich an seinem Ziel ankommt. Voraussetzung dafür ist es, das Ziel zu kennen.

Deshalb werden LBS bei Notrufen verwendet. Die Idee dahinter ist, wenn jemand den Notruf wählt, in Deutschland ist das die Nummer „112“ und Amerika „911“, wird der Leitstelle, welche den Anruf entgegen nimmt der Standort des Anrufers übermittelt. Dies ist nur bei Notrufen von mobil Telefonen möglich. Noch während des Anrufes kann ein Rettungswagen oder ein Einsatzwagen der Feuerwehr in Richtung des Anrufers aufbrechen. Der Anrufer hat so mehr Zeit seinen Standort detailliert mitzuteilen.

Bereits 1996 wurde in Amerika ein Gesetz verabschiedet welches den Mobilfunkanbieter dazu verpflichtet den ungefähren Standort des Anrufers bei einem Notruf mit zu übermitteln.  2003 wurde für Europa ein ähnliches Gesetzt verabschiedet. Das amerikanische Gesetz wurde 2001 überarbeitet und die Genauigkeit des Standortes muss nun zwischen 50-300 m liegen.  

Aus einer Quelle von 2004 lässt sich entnehmen, dass in den USA ca. 33\% (170.000 täglich) und in Europa 50-70\% (220.000 täglich) der Notruf mit Hilfe eines mobil Telefons getätigt werden. Man geht davon aus, dass mit dieser Technik ca. 5000 Menschenleben jährlich gerettet werden können.

Zur Ortung wird kein GPS- Modul benötigt, über sogenannten ID-Zellen kann der Standort des Anrufes auf bis zu 100m genau bestimmt werden. Die einzelnen Techniken werden in Kapitel XX genauer erläutert werden. 


\textbf{ \underline{Nutzung im Kommerziellen Umfeld } }

\paragraph{Praktische Einsatzgebiete}
Nach einer Goldmedia-Analyse~\cite[S.9]{goldmedia:lbs} verteilten sich die deutsche LBS-Marktstruktur 2014 auf 15 unterschiedliche Gebiete.\\
In der Studie werden folgende Punkte unterschieden:
\begin{itemize}
	\item Tourismus
	\item Beförderung und Verkehr
	\item Navigation und Maps
	\item Gastronomie
	\item Couponing und Einkauf
	\item Social
	\item Taxi
	\item Sport
	\item Augmented Reality
	\item Allgemeine Informationen
	\item Carsharing
	\item Gaming
	\item Gesundheit
	\item Media
	\item Sonstiges
\end{itemize}
Ganz offensichtlich ist diese Unterteilung vielschichtiger als die von Allan J Brimicombe und Chao Li. Es werden jeweils andere Schwerpunkte gesetzt. Es gibt jedoch auch Gemeinsamkeiten.

\paragraph{Gemeinsamkeiten und Unterschiede}
Navigation ist ein wichtiger Punkt in beiden Übersichten. Den Standort anzuzeigen bzw. den Nutzer zu navigieren ist eine der ersten Anwendungsbereiche von LBS.