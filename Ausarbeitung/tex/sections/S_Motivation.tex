\subsection{Motivation}

Historisch gesehen war die Information über den eigenen Standort schon immer von großer Bedeutung. In Verbindung mit Daten, die auf den Standort bezogen sind, war der Mehrwert enorm. So konnte man z.B. in der Seefahrt mit einem durch Sterne bestimmtem Standort in Verbindung mit einer Karte die Reiseroute so bestimmen, dass man Land erreicht bevor die Vorräte ausgehen.

Dieses noch recht weit hergeholte Beispiel über die Nutzung von Standort-bezogenen Daten konnte mit Hilfe von Technik auf die Informatik übertragen werden. Damit wurde der Begriff Location-based Services geprägt. 
%Fachliteratur über Location-based Services existiert auch schon seit mehr als 10 Jahren. 

Dem Endverbraucher waren solche Services zunächst einmal nur über spezielle Geräte, wie Navigationssystem möglich. In den letzten Jahren wurden milliardenfach Smartphones mit GPS-Modulen und mobilem Internet verkauft. Solch ein Smartphone besitzen nun eine Mehrzahl der in Deutschland lebenden Menschen. Durch die gerade erwähnten technischen Gegebenheiten der Smartphones ist die Nutzung von Location-based Services für den Nutzer ein Kinderspiel geworden. 

Diese Arbeit beschäftigt sich mit einer theoretischen Erarbeitung zu Location-based Services und eine prototypische Implementierung einer LBS Smartphone Applikation.

\newpage