\subsection{Architektur}
Die App ist streng nach dem Konzept des Model-View-Controller aufgebaut.
\\
\textbf{...Exkurs zu Model-View-Controller...
}\\
Einfach gesagt, besteht die App lediglich aus einer index.html Seite. Diese wird jedoch mithilfe von Javascript ständig mit neuem Inhalt (einer neuen View) gefüllt.
\\
Für jede dargestellte Seite der App existiert eine View. Diese besteht ebenfalls wieder aus einer einfache HTML-Seite, die allerdings über einen extra dafür definierten Controller mittels Variablenzuweisung verändert werden kann. Dieser Controller implementiert auch die Methoden, die dann z.B. bei einem Button-Druck ausgeführt werden können.
\\
In jedem Controller besteht die Möglichkeit auf sogenannte Services zuzugreifen. Diese bilden das Model in unserer Model-View-Controller-Darstellung. Sie enthalten die Daten, indem Sie sowohl auf den lokalen Speicher der App als auch auf die Daten des Geräts zugreifen können. Außerdem können an dieser Stelle auch selbst definierte Werte und damit verbundene Funktionen hinterlegt werden, die jeder Controller aufrufen kann.
\\
Um zwischen den einzelnen View wechseln zu können, wird immer ein aktueller Status der App gespeichert. Dieser Status entscheidet, welche View und welcher dazugehörige Controller in dem Moment verwendet werden. Über den Status der App kann man ganz leicht zwischen den Views unterscheiden und jeder Status hat seinen eigenen Historienstapel. Dadurch lassen sich vergangene Klicks leicht rückgängig machen.
\\
\textbf{..Detaillierte Darstellung der einzelnen Views in Kombination mit ihren jeweiligen Controllern + Services - graphisch aufbereitet!..}