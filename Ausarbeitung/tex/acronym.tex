%% Hier alle Abkürzungen eintragen und mit \ac{Abk.} benutzen.
%% ac{ABK} setzt die Abkürzung automatisch 
%% acl{Abk} schreibt die lange Version
%% wenn an einen /ac Befehl ein p angehängt wird /acp wird die Pluralform
% verwendet

 \acro{API}{Application programming interface} 
 \acro{App}{Applikation}
 \acro{AG}{Aktien Gesellschaft}
 \acro{BLE}{Bluetooth Low Energy}
 \acro{CSS}{Cascading Style Sheets}
 \acro{ESA}{Europäische Raumfahrt Behörde}
 \acro{FBCB2}{Force Twenty One Battle Command Brigade and Below}
 \acro{GPS}{Global Positioning System}
  \acro{HTML}{Hypertext Markup Language}
 \acro{Km}{Kilometer}   
  \acro{LBS}{Location-based Services}
 \acro{PLZ}{Postleitzahl}
 \acro{POI}{Points of Interest}
 \acro{U.S.} {United States}
 \acro{USA}{Untied States of America}
 \acro{UUID}{Universally Unique Identifier}
 \acro{VLR}{Visitor Location Register} 
 \acro{WLAN}{Wireless Local Area Network}










