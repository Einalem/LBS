\newpage
\section{Fazit}


Mit dieser Studienarbeit wurden die theoretischen Grundlagen zu Location-based Services erarbeitet. 
Praxisbezogene Anwendungsfälle für LBS wurden aufgezeigt und beschrieben.

Ein Anwendungsfall wurde in dieser Studienarbeit als Smartphone-App prototypisch umgesetzt. 
Es wurde dabei gezeigt, dass eine Umsetzung von LBS mit den Technologien Apache Cordova, Ionic, HTML5, CSS und Angular-JS möglich ist. 

Mit der Entscheidung für diese Technologien, sind einige Vorteile entstanden. Hierzu zählt, dass die Vorkenntnisse aus diversen Vorlesungen wie Web Engineering genutzt werden konnten und somit eine schnelle Einarbeitung in das Thema mobile Applikation stattfand. Des Weiteren ist es gelungen, eine App zu entwickeln, welche wie eine native App aussieht. Von Vorteil war auch die einfache Nutzung der Hardware des Smartphones durch Plugins des Cordova Frameworks. Als Nachteil ist zu nennen, dass der Aufwand für die Installation und Aufsetzung der App hoch war.

Im Detail wurde auch die Nutzung von externem Kartenmaterial und die Erweiterung der Positionsbestimmung um iBeacons umgesetzt.
Resultierend aus der vorhergegangenen Analyse wurde Openstreetmap für den Prototypen gewählt, was sich auch im Nachhinein betrachtet als sehr gute Wahl für studentische Projekte bestätigt hat. Des Weiteren konnte mit der Erweiterung der Positionsbestimmung um iBeacons eine deutlich verbesserte Präzision erzielt werden. 

Abschließend ist zu nennen, dass die Smartphone-App erfolgreich umgesetzt wurde und damit gezeigt wurde, dass sich die Technologien Apache Cordova, Ionic, HTML5, CSS und Angular-JS gut für LBS eignen.


\subsection[Ausblick]{Ausblick
 \\ \textnormal{\small{\textit {Gemeinsam verfasst}}}}
Im Verlauf der Arbeit wurde eine kleine aber demonstrative Applikation entwickelt, die die Grundzüge der Positionsbestimmung via GPS, die Erweiterung der Technologie über IBeacons und auch die Nutzung von Kartenmaterial in diesem Zusammenhang aufzeigt. Die Applikation bietet außerdem auch weitere Forschungsbereiche wie zum Beispiel die Bestimmung der Höhe über dem Meeresspiegel eines Geräts oder die Unterstützung des Betriebssystems iOS.
\\
Die Bedeutung der LBS ist groß. Applikationen, die momentan häufig als kleine Spielereien im Umlauf sind werden aber wohl schnell auch in kommerziellen Bereichen weiterentwickelt werden. Die Risiken, die mit der ständigen Ermittelbarkeit der eigenen Position verbunden sind, lassen sich dabei nicht abstreiten. Wie hilfreich wäre es für einen Einbrecher zu wissen, dass der Herr des Hauses momentan im Cafe nebenan sitzt? Bei der Entwicklung von unterstützender Software auf Basis der LBS wird daher in Zukunft noch viel mehr auf Datenschutz und Sicherheit geachtet werden müssen. Die Vorteile, die LBS mit sich bringen können werden aber wohl für viele über die Nachteile überwiegen. Vielleicht werden wir bald schon so gewohnt freizügig mit unseren Positionsdaten umgehen wie viele schon jetzt mit ihrer Telefonnummer, ihrer E-Mail- oder Privatadresse.