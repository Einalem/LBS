\section{Location based Services in der Praxis}
	\subsection{Anwendungsbereiche}
Location Based Services, also mobile, positionsbezogene Dienste haben allgemein ein sehr breites Einsatzgebiet.\\
\paragraph{Theoretische Einsatzgebiete}
Der Autoren Allan J Brimicombe und Chao Li unterscheiden in ihrem Buch "'Location-Based Services and Geo-Information Engineering"' ~\cite[S.132]{brimicombe_li:application_area} zehn verschiedene Einsatzgebiete:
\begin{itemize}
	\item Navigation\\
Navigation ist die gezielte Führung des Nutzers von Punkt A nach Punkt B. Einige Geräte bieten auch eine Echtzeit-Analyse an.
	\item Wegfindung\\
Bei der Wegfindung hingegen liegt der Fokus auf dem Finden möglicher Wege, d.h. sie dient der allgemeinen Orientierung des Nutzers.
	\item Echtzeit-Verfolgung\\
Verfolgungs- auch Tracking-Systeme genannt, dienen der Echtzeitanalyse des Nutzerstandorts, um diesem z.B. das Finden von Freunden in der näheren Umgebung zu erleichtern.
	\item Elektronischer Handel\\
Bei Anwendungen aus dem Bereich des elektronischen Handel, auch E-Commerce genannt, handelt es sich um werbende Produkte, die dem Nutzer auf Basis seiner Position ortsspezifische Angebote eröffnen.
	\item User-solicited Informations (vom Nutzer gewünschte Informationen)\\
Unter diese Kategorie fallen alle Anwendungen, die vom Nutzer für den geschäftlichen oder sozialen Gebrauch genutzt werden. Beispiele dafür sind: Wetterprognosen, Zugverspätungen und Filmvorführungen.
	\item Ortsgebundene Tarife
	\item Fulfilment
	\item Koordination
	\item Kunstvoller Ausdruck
	\item Mobile Spiele
\end{itemize}

\paragraph{Praktische Einsatzgebiete}
Nach einer Goldmedia-Analyse~\cite[S.9]{goldmedia:lbs} verteilten sich die deutsche LBS-Marktstruktur 2014 auf 15 unterschiedliche Gebiete.\\
In der Studie werden folgende Punkte unterschieden:
\begin{itemize}
	\item Tourismus
	\item Beförderung und Verkehr
	\item Navigation und Maps
	\item Gastronomie
	\item Couponing und Einkauf
	\item Social
	\item Taxi
	\item Sport
	\item Augmented Reality
	\item Allgemeine Informationen
	\item Carsharing
	\item Gaming
	\item Gesundheit
	\item Media
	\item Sonstiges
\end{itemize}
Ganz offensichtlich ist diese Unterteilung vielschichtiger als die von Allan J Brimicombe und Chao Li. Es werden jeweils andere Schwerpunkte gesetzt. Es gibt jedoch auch Gemeinsamkeiten.

\paragraph{Gemeinsamkeiten und Unterschiede}
Navigation ist ein wichtiger Punkt in beiden Übersichten. Den Standort anzuzeigen bzw. den Nutzer zu navigieren ist eine der ersten Anwendungsbereiche von LBS.
	\subsection{Typen von Location based Services (proaktiv und reaktiv)}
Typen
	\subsubsection{Typen Teil 1}
...
	\subsection{Location based Services auf mobilen Endgeräten}
Beispiele
	\subsubsection{Aufzählen vieler Anwendungsbeispiele mit Erläuterung des Nutzens}
...
	\subsubsection{Umsetzungsmöglichkeiten für die Beispiele nennen}
...