\subsection{Technologien und Entscheidungen}

\subsubsection{Cordova Phonegap}

\subsubsection{HTML5}

\subsubsection{CSS}

\subsubsection{JS}

\subsubsection{Kartenmaterial}

Kartenmaterial im Browser bzw. der Hybrid-App ist ein essentieller Bestandteil von Location based Services. Durch eine Positionsbestimmung alleine erhält man nur Daten die für den Nutzer nicht anschaulich sind. Diese liegen normalerweise als geografische Koordinaten vor, die in geografischer Breite und geografischer Länge angegeben werden. Eine Beispielposition soll die Bedeutung von Kartenmaterial für den Nutzer von Location based Services verdeutlichen.

Als Beispiel hierfür wurde die Position der DHBW Mannheim in der Coblitzallee gewählt. Hierbei werden die geografischen Koordinaten, eine Adresse und ein Kartenausschnitt in einer Tabelle gegenübergestellt. Siehe hierzu Tabelle \ref{BedeutungVonKartenmaterial}.

\begin{table}[htbp]
\begin{center}
\begin{tabular}{|p{4.75cm}p{4.75cm}p{4.75cm}|} 
	\hline
		\rowcolor{black} \textcolor{white} { \textbf{Geographische Koordinaten} } & \textcolor{white}{\textbf{Adresse}} & \textcolor{white}{\textbf{Kartenausschnitt}}\\ 
		\rowcolor[gray]{.75}  49$^\circ$28'27.6\grqq N 8$^\circ$32'03.9\grqq E & Duale Hochschule Baden-Württemberg Mannheim \newline 
Coblitzallee 1-9 \newline 
68163 Mannheim \newline (Neuostheim) & Kartenausschnitt\newline 
\includegraphics[width=0.3\textwidth]{ref/images/KartenmaterialKlein.png} \\ 
\hline
	\end{tabular}
\end{center}
\caption{Bedeutung von Kartenmaterial} \label{BedeutungVonKartenmaterial}
\end{table}

In der Tabelle sind verschiedenen Ortsdaten zur Verfügung gestellt, die alle Vor- und Nachteile aufweisen.

Die Geographischen Koordinaten geben die Position am genausten an, sind für fast keine Nutzer einer App von Bedeutung. 

Die Adresse ist im Alltag am geläufigsten und somit für Nutzer am verständlichsten. Allerdings ist die Angabe nicht so genau, wie die Geographischen Koordinaten. Denn die Angabe Hausnummer 1-9 gibt einen relativ großen Bereich an.

Die Vorteile eines Kartenausschnitts sind, dass die Detaillierung vom Nutzer angepasst werden kann. Des Weiteren werden viele grafische Informationen angezeigt, wie zum Beispiel der eigene Standort, an denen sich ein Nutzer Orientieren kann. Der Nachteil dieser Variante ist, dass die Kartenausschnitte die Zuhilfenahme von externen Quellen und einem erhöhten TODO: Programmieraufwand mit sich bringen.


TODO: Quelle finden
Auf Smartphones gehört Kartenmaterial und dessen Integration in Apps mittlerweile zum Standard, an welchen sich Nutzer gewöhnt haben. Aus diesem Grund sollte auch Kartenmaterial in die Location based Services App integriert werden, welche die Autoren bei dieser Studienarbeit entwickeln. 

Mögliche Quellen für das Kartenmaterial sind \glqq Google Maps\grqq, \glqq Bing Maps\grqq  und \glqq Open Street Maps\grqq.


\textbf{Anforderungen an das Kartenmaterial}

Die Anforderungen an interaktives Kartenmaterial bezüglich der in dieser Studienarbeit entwickeltem Projekt lassen sich in zwei Gruppen einteilen, die funktionalen und nichtfunktionalen Anforderungen.

Die \underline{nichtfunktionalen Anforderungen} sind:
\begin{enumerate}
\item Kostenlose Abfragen
\item Ohne Account nutzbar
\item Gute Dokumentation mit Codebeispielen
\item Zukunftssicherheit
\end{enumerate}

Die \underline{funktionalen Anforderungen} an das interaktive Kartenmaterial sind:
\begin{enumerate}
\item JavaScript API
\item Unterstütze Browser
\item Eigenen Standort anzeigen
\item Markierungen auf der Karte setzen
\item Markierungen bündeln (optional)
\end{enumerate}

Bevor \glqq Google Maps\grqq, \glqq Bing Maps\grqq  und \glqq Open Street Maps\grqq bezüglich der Anforderungen untersucht werden, müssen diese genauer spezifiziert werden.

Zuerst widmen wir uns den nichtfunktionalen Anforderungen.
\begin{enumerate}
\item Kostenlose Abfragen \\
Da es sich bei der Implementierung um einen Prototypen für diese Studienarbeit handelt und dieser nicht kommerziell verwendet werden soll, sollen auch die Abfragen (map-loads) kostenlos sein. Zudem sollten genug kostenlose Abfragen zur Verfügung stehen. Bei 3 Entwicklern und Tests über die Dauer der Studienarbeit (8-9 Monate) darf das Kontingent der kostenlosen Abfragen nicht aufgebraucht sein.

\item Ohne Account nutzbar\\
Die Nutzung ohne Account vereinfacht den Einstieg für das Kartenmaterial und sollte deshalb gewährleistet sein. Zudem würden dann alle Teammitglieder dieser Studienarbeit mit dem selben Account eines Teammitglieds arbeiten. 

\item Gute Dokumentation mit Codebeispielen\\
Eine gute Dokumentation der API des Kartenmaterials mit vielen Codebeispielen erleichtert den Einstieg deutlich. Da alle Teammitglieder dieser Studienarbeit noch keine Erfahrung mit Kartenmaterial haben ist dies eine wichtige Anforderung. 

\item Zukunftssicherheit\\
Da diese Arbeit nicht nur den aktuellen Stand der Technik abbilden soll, sondern auch einen Ausblick geben soll ist es wichtig, dass auch beim Kartenmaterial auf die Zukunftssicherheit geachtet wird.
\end{enumerate}



TODO: Überleitung

\begin{enumerate}
\item JavaScript API\\
Eine JavaScript API ist essentiell wichtig, da der Prototyp mit HTML, CSS und Java Script entwickelt wird. 

\item Unterstütze Browser\\
Android ??, IOS, 

\item Eigenen Standort anzeigen\\
Der eigene Standort muss grafisch auf einer interaktiven Karte angezeigt werden. Das zentrieren des Kartenmaterials auf den eigenen Standort soll auch möglich sein.

\item Markierungen auf der Karte setzen\\
Eigenen Markierungen müssen auf der Karte gesetzt werden könne. Dies muss grafisch erfolgen, denn es ist für den Prototypen besonders wichtig zu veranschaulichen, wo sich das gesuchte Ziel befindet. 

\item Markierungen bündeln (optional)\\
Markierungen auf der Karte sollten gebündelt werden, wenn der Zoom-Faktor zu klein wird. Dies soll der Übersichtlichkeit bei vielen Markierungen auf der Karte dienen. Hierbei soll der Radius, in dem Markierungen gebündelt werden eingestellt werden könne.

\end{enumerate}



\textbf{\underline{Google Maps}}\\
Seit 2005 gehört zu dem Produktportfolio von dem Suchmaschinenriesen Google ein Internet Kartendienst namens Google Maps. 
Google Maps gehört zu den verbreitetsten und erfolgreichsten Internet Kartendiensten der Welt. \cite[Lexikon Google Maps]{itwissen}\cite[S.88]{gruber2015}

Zunächst wird Google Maps anhand der nichtfunktionalen Anforderungen bewertet, danach werden die funktionalen Anforderungen betrachtet.

\textbf{Nichtfunktionale Anforderungen}
\begin{enumerate}
\item Kostenlose Abfragen \\
Die Google Maps API steht generell als kostenloser Dienst zur Verfügung. Dieser darf in Webseiten, sowie mobile Apps eingebaut werden. Als Voraussetzung für die kostenlose Nutzung gilt allerdings, dass die eigene Webseite oder mobile App für alle Endnutzern kostenlos ist und öffentlich zugänglich sein muss.\\
Unter der kostenlosen Lizenz dürfen bis zu 25.000 Kartenladevorgänge pro Tag erfolgen. Ein Kartenladevorgang ist als Initiales Laden der Karte Definiert. Das bedeutet, dass Nutzerinteraktion mit der Karten nicht als erneuter Ladevorgang gewertet wird. 
In den Nutzungsbedingungen von Google wird darauf verwiesen, dass eine Sperre von mehr als 25.000 Abfragen pro Tag erst dann durchgeführt wird, wenn die Begrenzung mehr als 90 Tage in Folge überschritten werden sollte.
Diese Auslegung erscheint sehr nutzerfreundlich. Für diese Studienarbeit sollten die 25.000 Abfragen pro Tag in jedem Fall völlig ausreichen. \cite[Nutzungsbedingungen]{googlemaps}\cite[Lizenzierung]{googlemaps}
https://developers.google.com/maps/licensing?hl=de

\item Ohne Account nutzbar\\
Seit dem die Google Maps JavaScript API in der Version 3 vorliegt ist die Nutzung ohne einen Schlüssel möglich. Das bedeutet, dass keine Registrierung bzw. Anmeldung zum nutzen nötig ist. Diese Praxis erleichtert den Einstieg in Google Maps, da man sofort starten kann.
http://googlegeodevelopers.blogspot.de/2010/03/introducing-new-google-geocoding-web.html

\item Gute Dokumentation mit Codebeispielen\\
Google Maps bietet ein Entwicklerhandbuch für die JavaScript API v3, dass sowohl ausführlich ist, als auch viele Beispiele bietet. Zudem ist das Benutzerhandbuch in deutsch abrufbar, was das erarbeiten und nachlesen vereinfacht. 
Zu dem ausführlichen Entwicklerhandbuch gibt es nochmals fast 150 Beispiele, zu denen die jeweils passende Output (Karte) angezeigt wird.\cite[Documentation]{googlemaps} \\
Neben der Dokumentation von Google selbst gibt es auch andere Tutorials im Netz. Eines davon ist von der Webseite www.w3schools.com, dass versucht von Grund auf eine Einführung in Google Maps zu geben. 
https://developers.google.com/maps/documentation/javascript/examples/?hl=de

\item Zukunftssicherheit\\
Der Internet Kartendienst ist sehr erfolgreich und weltweit verbreitet, was darauf schließen lässt, dass Google selbst ein großes Interesse daran hat diesen auch in Zukunft weiter zu führen und weiter zu entwickeln. 
Man sollte sich allerdings bewusst sein, dass die Anzahl der kostenlosen Abfragen sich ändern kann, oder dass der Dienst irgendwann generell kostenpflichtig wird.

\end{enumerate}


\textbf{funktionale Anforderungen}
\begin{enumerate}
\item JavaScript API\\
Mit der Google Maps JavaScript API Version 3 bietet Google eine JavaScript API die sich sehr leicht in Webseiten integrieren lässt. 

Bei einer HTML-Webseite muss lediglich ein Script Tag eingefügt werden, in dem als Quelle (src) die Google Maps API zu finden ist. Vergleiche Listing Zeile 1. Danach können mit JavaScript Karten von Google Maps erstellt werden. Vergleiche Listing Zeile 2 - 12.
\begin{lstlisting}
<script src="https://maps.googleapis.com/maps/api/js?v=3.exp"></script>
    <script>
		var map;
		function initialize() {
		  var mapOptions = {
		    zoom: 8,
		    center: new google.maps.LatLng(-34.397, 150.644)
		  };
		  map = new google.maps.Map(document.getElementById('map-canvas'),
	      mapOptions);
		}
		google.maps.event.addDomListener(window, 'load', initialize);
    </script>
\end{lstlisting} \cite[Codebeispiel Simple Map]{googlemaps}


\item Unterstütze Browser\\
TODO

\item Eigenen Standort anzeigen\\
Den eigenen Standort kann man, sofern dieser bestimmt werden konnte (Listing Zeile 1 - 5), mit der Google Maps API auf der Karte anzeigen. Dies kann man zum Beispiel mit einer Informationsbox, die man mit \glqq new google.maps.Infowindow \grqq auf dem eigenen Standort erstellt erfolgen. (Listing Zeile 7 - 11) Zur besseren Veranschaulichung wird die Karten dann noch auf die Eigene Position zentriert (Listing Zeile 13)

\begin{lstlisting}
  // Try HTML5 geolocation
  if(navigator.geolocation) {
    navigator.geolocation.getCurrentPosition(function(position) {
      var pos = new google.maps.LatLng(position.coords.latitude,
                                       position.coords.longitude);

      var infowindow = new google.maps.InfoWindow({
        map: map,
        position: pos,
        content: 'Location found using HTML5.'
      });

      map.setCenter(pos);
    }
\end{lstlisting} \cite[Codebeispiel Geolocation]{googlemaps}

\item Markierungen auf der Karte setzen\\
Markierungen könne mit der API sehr leicht gesetzt werden. Neben der Position der Markierung muss noch eine Referenz auf die Google Maps Karte, sowie ein Name bei der Erstellung angegeben werden. Vergleiche Listing.

\begin{lstlisting}
  var marker = new google.maps.Marker({
      position: new google.maps.LatLng(-25.363882,131.044922),
      map: map,
      title: 'Hello World!'
  });

\end{lstlisting} \cite[Codebeispiel Simple Markers]{googlemaps}

Den Markierungen auf der Karte können allerdings auch Bilder mit dem Attribut \glqq icon \grqq zugeordnet werden. Des Weiteren kann man mit dem Attribut \glqq draggable \grqq einstellen, ob man die Markierung verschieben kann oder nicht.

\item Markierungen bündeln (optional)\\
Markierungen können in der Standard JavaScript API in Version 3 nicht gebündelt werden. 
Mit einer zusätzlichen Bibliothek  von Google kann dieser Funktionalität allerdings ergänzt werden. Die Bibliothek heißt \glqq google-maps-utility-library-v3 \grqq. 
Mit Hilfe dieser Bibliothek kann ein \glqq Markercluster \grqq erstellt werden, dass Markierungen bei einer gewissen Zoomstufe bündelt. Hierbei wird dem Markercluster ein Array der Markierungen, eine Refernz auf dei Karte, sowie Markercluster-Einstellungen.
TODO: Formulierung an Listing anpassen
\begin{lstlisting}
var mcOptions = {gridSize: 50, maxZoom: 15};
var markers = [...]; // Create the markers you want to add and collect them into a array.
var mc = new MarkerClusterer(map, markers, mcOptions);
\end{lstlisting}

https://googlemaps.github.io/js-marker-clusterer/docs/examples.html

\end{enumerate}





\textbf{\underline{Bing Maps}}\\
TODO: Einleitung

\textbf{Nichtfunktionale Anforderungen}
\begin{enumerate}
\item Kostenlose Abfragen \\
Bing Maps bietet für öffentlich zugängliche Webseiten, sowie für mobile Apps für Konsumenten ein kostenloses Kontingent von 125.000 Transaktionen pro Jahr. Will man dieses Kontingent von Transaktionen überschreiten, werden Kosten fällig.

\item Ohne Account nutzbar\\
Bing Maps ist nicht ohne einen Account nutzbar. Bevor man mit der dazugehörigen API entwickeln kann ist es nötig zuerst eine Microsoft ID anzulegen, mit der man dann wiederum einen Entwickler Key für Bing Maps anfordern kann. Ein schneller Einstieg ist auf Grund von Registrierungen nicht möglich. Zudem gibt es unterschiedliche Lizenzmodelle, die zwischen Webseite und mobile App unterscheiden, was zur Folge hat, dass man für eine neue Plattform einen neuen Key benötigt.

\item Gute Dokumentation mit Codebeispielen\\
Bing Maps bietet eine ausführliche Dokumentation für die Bing Maps AJAX Control Version 7.0. Dabei werden die Klassen beschrieben und fast jede mit einem Beispiel verdeutlicht. 
Zusätzlich gibt es über 200 Codebeispiele, die man direkt im Browser ausprobieren und editieren kann. Diese Beispiele kann man auch ohne Account (Key) nutzen.

\item Zukunftssicherheit\\
TODO

\end{enumerate}



\textbf{funktionale Anforderungen}
\begin{enumerate}
\item JavaScript API\\
\begin{lstlisting}
<script type="text/javascript" src="http://ecn.dev.virtualearth.net/mapcontrol/mapcontrol.ashx?v=7.0"></script>
      <script type="text/javascript">
      var map = null;
      function getMap()
      {
          map = new Microsoft.Maps.Map(document.getElementById('myMap'), {credentials: 'Your Bing Maps Key'});
      }   
      </script>
\end{lstlisting} \cite[Codebeispiel CreateMap1]{bingmaps}
\item Unterstütze Browser\\
TODO
\item Eigenen Standort anzeigen\\
\begin{lstlisting}
map.entities.clear(); 
var geoLocationProvider = new Microsoft.Maps.GeoLocationProvider(map);  
geoLocationProvider.getCurrentPosition(); 
displayAlert('Current location set, based on your browser support for geo location API');
\end{lstlisting}\cite[Codebeispiel GetUserLocation1]{bingmaps}
\item Markierungen auf der Karte setzen\\
\begin{lstlisting}
map.entities.clear(); 
var pushpin= new Microsoft.Maps.Pushpin(map.getCenter(), null); 
map.entities.push(pushpin); 
pushpin.setLocation(new Microsoft.Maps.Location(47.5, -122.33)); 
displayAlert('Pushpin Location Updated to ' + pushpin.getLocation() + '. Pan map to location, if pushpin is not visible');
\end{lstlisting}\cite[Pushpins7]{bingmaps}
\item Markierungen bündeln (optional)\\
TODO: ix
\end{enumerate}


Kostenlose Abfragen für education use
Viele unterschiedliche Lizenzmodelle 
125.000 Abfragen pro Jahr

Man muss allerdings einen Bing Maps Account erstellen

http://www.microsoft.com/maps/Licensing/licensing.aspx\#StepTitle


Ausführliche Dokumentation mit vielen Beispielen
+ für in Browser Tests auf der Webseite
https://www.bingmapsportal.com/Isdk/AjaxV7\#CreateMapWithMapOptions1

https://msdn.microsoft.com/en-us/library/gg427610.aspx

\textbf{funktionale Anforderungen}
JavaScript gegeben

Supported Browsers
https://msdn.microsoft.com/en-us/library/gg427618.aspx


\textbf{\underline{Open Street Maps}}

Kostenlos und unbegrenzt viele Abfragen

ohne Account nutzbar

Gute Dokumentation mit vielen Beispielen leider nicht gegeben




