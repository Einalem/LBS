\section{Anwendungsfälle für LBS}

Aufzählen wie in SE (Nutzer Potentiale, welches Nutzerproblem wird befriedigt, wie ist Konkurrenz)

Wecker
Restaurantfinder
Navigation
Freunde finden
Reiseführer
Spiele


\subsection{Historie von LBS}
Bedeutung von Location Based Services in der heutigen Zeit. Erfunden wurden LBS von der U.S. abwehr behörde. Hierfür wurde das System Navstar entwickelt. Mit Hilfe von Sattelitten kann die Position bis auf wenige Meter bestimmt werde. Bekannt geworden ist diese System unter dem Namen Global Positioning System (GPS). Nutzer dieses System war das Militär. 1980 entschied man sich dazu, dass System der Öffentlichkeit bereitzustellen. Ziel dieses Schrittes war es, Fortschritte in der Entwicklung von Satelliten Systemen zu machen.

Die Europäische Union entwickelte daraufhin mit der Europäischen Raumfahrt Behörde (ESA) einen eigenes System namens Galileo. Galileo verwendet ähnliche Frequenzen wie GPS, würde die UN oder die USA diese Signale blockieren, wären beide Systeme gestört, was politisch von hoher Bedeutung ist.




Location Based Services, also mobile, positionsbezogene Dienste haben allgemein ein sehr breites Einsatzgebiet.\\
\paragraph{Theoretische Einsatzgebiete}
Der Autoren Allan J Brimicombe und Chao Li unterscheiden in ihrem Buch "'Location-Based Services and Geo-Information Engineering"' ~\cite[S.132]{brimicombe_li:application_area} zehn verschiedene Einsatzgebiete:
\begin{itemize}
	\item Navigation\\
Navigation ist die gezielte Führung des Nutzers von Punkt A nach Punkt B. Einige Geräte bieten auch eine Echtzeit-Analyse an.
	\item Wegfindung\\
Bei der Wegfindung hingegen liegt der Fokus auf dem Finden möglicher Wege, d.h. sie dient der allgemeinen Orientierung des Nutzers.
	\item Echtzeit-Verfolgung\\
Verfolgungs- auch Tracking-Systeme genannt, dienen der Echtzeitanalyse des Nutzerstandorts, um diesem z.B. das Finden von Freunden in der näheren Umgebung zu erleichtern.
	\item Elektronischer Handel\\
Bei Anwendungen aus dem Bereich des elektronischen Handel, auch E-Commerce genannt, handelt es sich um werbende Produkte, die dem Nutzer auf Basis seiner Position ortsspezifische Angebote eröffnen.
	\item User-solicited Informations (vom Nutzer gewünschte Informationen)\\
Unter diese Kategorie fallen alle Anwendungen, die vom Nutzer für den geschäftlichen oder sozialen Gebrauch genutzt werden. Beispiele dafür sind: Wetterprognosen, Zugverspätungen und Filmvorführungen.
	\item Ortsgebundene Tarife
	\item Fulfilment
	\item Koordination
	\item Kunstvoller Ausdruck
	\item Mobile Spiele
\end{itemize}

\paragraph{Praktische Einsatzgebiete}
Nach einer Goldmedia-Analyse~\cite[S.9]{goldmedia:lbs} verteilten sich die deutsche LBS-Marktstruktur 2014 auf 15 unterschiedliche Gebiete.\\
In der Studie werden folgende Punkte unterschieden:
\begin{itemize}
	\item Tourismus
	\item Beförderung und Verkehr
	\item Navigation und Maps
	\item Gastronomie
	\item Couponing und Einkauf
	\item Social
	\item Taxi
	\item Sport
	\item Augmented Reality
	\item Allgemeine Informationen
	\item Carsharing
	\item Gaming
	\item Gesundheit
	\item Media
	\item Sonstiges
\end{itemize}
Ganz offensichtlich ist diese Unterteilung vielschichtiger als die von Allan J Brimicombe und Chao Li. Es werden jeweils andere Schwerpunkte gesetzt. Es gibt jedoch auch Gemeinsamkeiten.

\paragraph{Gemeinsamkeiten und Unterschiede}
Navigation ist ein wichtiger Punkt in beiden Übersichten. Den Standort anzuzeigen bzw. den Nutzer zu navigieren ist eine der ersten Anwendungsbereiche von LBS.


\subsection{Hauptnutzer von LBS}

LBS wurden erstmals vom amerikanischen Militär erfunden und genutzt. Nachdem die Services der Öffentlichkeit bereitgestellt wurden, führte dies zu immer mehr Anwendungsbereichen beispielsweise zur Lokalisierung von Notrufen. In Europa findet dies über die Rufnummer „112“ statt, in Amerika „911“.  Seit 1996 besteht in den USA eine Pflicht, den ungefähren Standort mitzusenden bei einem Notruf. 

Im Laufe der letzten Jahre wurden immer mehr Möglichkeiten geschaffen, mobil Telefone zu lokalisieren und den Standort für beispielsweise Informationsdarstellung zu nutzen. Damit ergibt sich der dritte große Anwendungsbereich von LBS, die kommerzielle Nutzung für privat Personen und Unternehmen.

Im folgenden Abschnitt werden die genannten Hauptnutzer von LBS genauer erläutert und aufgezeigt wofür LBS verwendet wird.


\underline{LBS im Umfeld des Militärs}

Den eigenen Standort zu kennen und beschreiben zu können ist nicht immer auf Anhieb möglich. Befindet man sich an einem Ort, bei dem viele Sehenwürdigkeiten oder bekannte Gegenstände wie beispielsweise Häuser, Parks, Straßen in der Nähe sind, fällt es einem meist einfacher den eigenen Standort einer anderen Person mittzuteilen, damit dieser einen findet.

Wesentlich schwieriger ist die Standortbestimmung wenn man sich an einem Ort befindet, der sehr allgemein ist und keine Besonderheiten bzw. identifikations Merkmale aufweist. Oft befindet sich das Militär an solchen schwer zu definierenden Orten. Einige Beispiele für solche Orte sind: Wüsten, Wälder, Berge und Gebirge. Vermutlich war dies einer der Hauptgründe ein System zur Bestimmung des Standortes zu entwickeln. 

Die Anwenudng FBCB2 ist ein Beispiel für die Verwendung von mehreren Standorten. Sie wird bereits seit 10 Jahren vom amerikanischen Militär eingesetzt. FBCB2 ist die Abkürzung für „Force-Twenty-One Battle Command Brigade and Below“. Die Anwendung ist bei Panzer Brigaden anführern im Einsatz. Auf einer Karte wird dem Nutzer angezeigt welche verbündeten Panzer in der Nähe sind. Zu jedem dieser Panzer werden einem weiter Informationen bereitgestellt. Mit diesem System braucht man keinen Kompass und keine Papier- Karte mehr um sich eine Überblick zu verschaffen. 

Im Kriegs Geschehen allgemein nimmt die Bedeutung von LBS stark zu. Genutzt wird dies Technik unter anderem bei Lenkraketen. Der Befehlshaber braucht nur die Koordinaten anzugeben und die Rakete berechnet den optimalen Weg zum Ziel. Essenziel wichtig ist dabei für die Rakete zu jedem Zeitpunkt im Flug zu wissen an welchen Standort sie sich befindet um gegebenenfalls die Geschwindigkeit oder Höhe anzupassen. 
