\newpage
\section{Fazit}


Mit dieser Studienarbeit wurden die theoretischen Grundlagen zu Location-based Services erarbeitet. 
Praxisbezogene Anwendungsfälle für LBS wurden aufgezeigt und beschrieben.

Ein Anwendungsfall würde in dieser Studienarbeit als Smartphone-App prototypisch umgesetzt. 
Es wurde dabei gezeigt, dass eine Umsetzung von LBS mit den Technologien Apache Cordova, Ionic, HTML5, CSS und Angular-JS möglich ist. 

TODO Vorteile und Nachteile dieser Variante zusammenfassen

Im Detail wurde auf die Nutzung von externem Kartenmaterial und die Erweiterung der Positionsbestimmung um iBeacons umgesetzt.
Resultierend daraus wurde festgestellt, dass sich Openstreetmap gut für studentische Projekte eignet. Des Weiteren konnte mit der Erweiterung der Positionsbestimmung um iBeacons eine deutlich verbesserte Präzision erzielt werden. 

Abschließend ist zu nennen, dass die Smartphone-App erfolgreich umgesetzt wurde und damit gezeigt wurde, dass die Technologien Apache Cordova, Ionic, HTML5, CSS und Angular-JS für LBS eignen.


\subsection[Ausblick]{Ausblick
 \\ \textnormal{\small{\textit {Verfasst von ??}}}}
 
