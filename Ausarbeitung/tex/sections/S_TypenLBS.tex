\subsection{Typen von LBS}

\textbf{Proaktiv, Reaktiv}

Literaturrecherche + viele Zitate fürs Quellenverzeichnis +Beacons (indoor)

\begin{table}[h]
	\centering
	\begin{tabular}{|p{16cm}|}\hline
		\textbf{Zitat ?:} \glqq LBSs can be classified into reactive and proactive LBSs. A reactive LBS is always explicitly activated by the user. \cite[S.3]{Kuepper2005} \\ \hline
		\textbf{Übersetzung:} LBSs können in reaktive und proaktive LBSs klassifiziert werden. \\ \hline
	\end{tabular}
\end{table}


\begin{table}[h]
	\centering
	\begin{tabular}{|p{16cm}|}\hline
		\textbf{Zitat ?:} \glqq Pull services, in contrast, mean that a user actively uses an application and, in this context, "pulls" information from the network. This information may be location-enhanced (e.g., where to find the nearest cinema) \cite[S.20]{Schiller2004} \\ \hline
		\textbf{Übersetzung:} TODO \\ \hline
	\end{tabular}
\end{table}


\begin{table}[h]
	\centering
	\begin{tabular}{|p{16cm}|}\hline
		\textbf{Zitat ?:} \glqq Proactive LBSs, on the other hand, are automatically initialized as soon as a predefined location event occurs, for example, if the user enters, approaches, or leaves a certain point of interest or if he approaches, meets, or leaves another target. \cite[S.3]{Kuepper2005} \\ \hline
		\textbf{Übersetzung:} Proaktive LBSs werden automatisch initialisiert, sobald ein zuvor definiertes Standortevent auftritt. Dies kann zum Beispiel sein, dass ein Nutzer eine Sehenswürdigkeit betritt, sich dieser annähert oder diese verlässt. Oder wenn der Nutzer sich einem anderen Ziel nähert, es trifft oder es verlässt. \\ \hline
	\end{tabular}
\end{table}

\begin{table}[h]
	\centering
	\begin{tabular}{|p{16cm}|}\hline
		\textbf{Zitat ?:} \glqq Push services imply that the user receives information as a result of his or her whereabouts without having to actively request it. The information may be sent to the user with prior consent (e.g., a subscription-based terror attack alert system) or without prior consent (e.g., an advertising welcome message sent to the user upon entering a new town). \cite[S.20]{Schiller2004} \\ \hline
		\textbf{Übersetzung:} TODO \\ \hline
	\end{tabular}
\end{table}


TODO: Some services such as a friend finder or date finder integrate both push and pull functionality.