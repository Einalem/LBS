\subsection[Anforderungen]{Anforderungen
 \\ \textnormal{\small{\textit {Verfasst von Melanie Hammerschmidt}}}}

Die Analyse im Vorfeld der Implementierung ergab die folgenden Anforderungen. Unterschieden wurden diese im Bezug auf Fachlichkeit und technischen Anspruch.
\begin{enumerate}
\item Funktionale Anforderungen
	\begin{enumerate}
		\item Grundeinstellungen für Spieler speichern
		\begin{enumerate}
			\item Name
			\item Spielradius		
			\item Startposition
		\end{enumerate}
		\item Aufgaben erstellen (entsprechend Startposition eines Spielers)
		\begin{enumerate}
			\item Erreichen von definierten Orten/Koordinaten
			\begin{enumerate}
				\item entsprechend gewähltem Radius
				\item steigender Schwierigkeitsgrad
			\end{enumerate}
			\item Erreichen einer gewissen Höhe
		\end{enumerate}
		\item Aufgaben spielen
		\begin{enumerate}
			\item Prüfung der aktuellen Position 
			\item Vergleich zwischen Aufgabe und aktueller Position
			\item ständige Rückmeldung (verbleibende Entfernung zum Ziel) an den Spieler
			\item Punkte neu berechnen (sammeln oder verlieren)
			\item Punkte in Highscore speichern
			\item Möglichkeit, das Spiel zu unterbrechen und später wieder zu starten
		\end{enumerate}
	\end{enumerate}
	\item Nicht-funktionale, technische Anforderungen
	\begin{enumerate}
		\item Plattformunabhängigkeit (vgl. Kapitel Technologien und Entscheidungen)
		\item Benutzbarkeit
		\begin{enumerate}
			\item übersichtliche Steuerung
			\item Selbstbeschreibend
			\item Erwartungskonformität
			\item Fehlertoleranz
		\end{enumerate}
		\item gute Animation der '"Wünschelroute'"
	\end{enumerate}
\end{enumerate}
